\documentclass[12pt,a4paper]{article}
\usepackage[utf8]{inputenc}
\usepackage[greek,english]{babel}
\usepackage{amsmath}
\usepackage{graphicx}
\usepackage{geometry}
\usepackage{booktabs}
\usepackage{float}
\usepackage{hyperref}
\usepackage{listings}
\usepackage{xcolor}

\geometry{margin=2.5cm}

\lstset{
    basicstyle=\ttfamily\small,
    breaklines=true,
    frame=single,
    language=Python,
    keywordstyle=\color{blue},
    commentstyle=\color{gray},
    stringstyle=\color{red}
}

\title{\textbf{Υπολογιστική Εργασία} \\ Αξιοπιστία Συστημάτων}
\author{
    Τζανέτης Σάββας \\ ΑΕΜ: 10889
    \and
    Βασίλειος Ζωίδης \\ ΑΕΜ: 10652
}
\date{\today}

\begin{document}

\selectlanguage{greek}
\maketitle
\newpage

\tableofcontents
\newpage

\section{Εισαγωγή}

Στην παρούσα εργασία μελετάται η αξιοπιστία ενός σύνθετου συστήματος που αποτελείται από 7 εξαρτήματα (C1-C7) με δομή σειράς και παράλληλων μπλοκ. Το σύστημα έχει την εξής τοπολογία:

\begin{center}
\texttt{C1 → [C2 || C3 || C4] → [C5 || C6] → C7}
\end{center}

όπου τα σύμβολα → υποδηλώνουν σύνδεση σε σειρά και || σύνδεση παράλληλα.

Η ανάλυση χωρίζεται σε δύο μέρη:
\begin{enumerate}
    \item \textbf{Ανάλυση χωρίς επιδιόρθωση (MTTR=0)}: Υπολογισμός ρυθμού αστοχίας ($\lambda$), αξιοπιστίας ($R$) και μέσου χρόνου μέχρι την αστοχία (MTTF)
    \item \textbf{Ανάλυση με επιδιόρθωση (MTTR>0)}: Υπολογισμός μέσου χρόνου μεταξύ αστοχιών (MTBF), μέσου χρόνου λειτουργίας (MUT), μέσου χρόνου επιδιόρθωσης (MTTR) και διαθεσιμότητας (A)
\end{enumerate}

\section{Παράμετροι Προσομοίωσης}

\subsection{Χαρακτηριστικά Εξαρτημάτων}

Τα χαρακτηριστικά των 7 εξαρτημάτων φαίνονται στον Πίνακα \ref{tab:components}.

\begin{table}[H]
\centering
\caption{Χαρακτηριστικά εξαρτημάτων}
\label{tab:components}
\begin{tabular}{@{}cccc@{}}
\toprule
\textbf{Εξάρτημα} & \textbf{MTTF (h)} & \textbf{Duty Cycle} & \textbf{MTTR (h)} \\ 
\midrule
C1 & 30 & 0.3 & 12 \\
C2 & 24 & 1.0 & 12 \\
C3 & 23 & 1.0 & 12 \\
C4 & 24 & 1.0 & 10 \\
C5 & 27 & 1.0 & 10 \\
C6 & 28 & 1.0 & 8 \\
C7 & 33 & 0.4 & 12 \\
\bottomrule
\end{tabular}
\end{table}

\subsection{Παράμετροι Προσομοίωσης Monte Carlo}

\begin{itemize}
    \item \textbf{Χρόνος μελέτης εξαρτημάτων (Tc)}: 100 ώρες
    \item \textbf{Χρόνος μελέτης συστήματος (Ts)}: 30 ώρες
    \item \textbf{Χρονικό βήμα (dt)}: 0.01 ώρες
    \item \textbf{Αριθμός προσομοιώσεων (N)}: 1000
\end{itemize}

\section{Μέθοδος Προσομοίωσης}

\subsection{Μοντέλο Συμπεριφοράς Εξαρτήματος}

Κάθε εξάρτημα μπορεί να βρίσκεται σε μία από τις εξής καταστάσεις:
\begin{itemize}
    \item \textbf{Κατάσταση 2 (Operational)}: Το εξάρτημα λειτουργεί κανονικά
    \item \textbf{Κατάσταση 1 (Non-operational - DC)}: Το εξάρτημα δεν λειτουργεί λόγω κύκλου εργασίας (duty cycle)
    \item \textbf{Κατάσταση 0 (Under repair)}: Το εξάρτημα έχει αστοχήσει και επιδιορθώνεται
\end{itemize}

\subsection{Μοντέλο Αστοχίας}

Η αστοχία κάθε εξαρτήματος ακολουθεί την εκθετική κατανομή με παράμετρο $\lambda = \frac{1}{\text{MTTF}}$. Η πιθανότητα αστοχίας σε διάστημα $dt$ δίνεται από:

\begin{equation}
P_{\text{fail}}(dt) = 1 - e^{-\lambda \cdot dt}
\end{equation}

Ο duty cycle (DC) επηρεάζει το πραγματικό MTTF ως:

\begin{equation}
\text{MTBF}_{\text{effective}} = \frac{\text{MTTF}}{\text{DC}}
\end{equation}

\subsection{Μοντέλο Επιδιόρθωσης}

Ο χρόνος επιδιόρθωσης ακολουθεί εκθετική κατανομή με μέσο χρόνο MTTR:

\begin{equation}
T_{\text{repair}} \sim \text{Exp}(\text{MTTR})
\end{equation}

\section{Αποτελέσματα - Ανάλυση χωρίς Επιδιόρθωση}

\subsection{Αξιοπιστία Εξαρτημάτων}

Τα αποτελέσματα για την αξιοπιστία των εξαρτημάτων στο χρόνο $T_c = 100h$ παρουσιάζονται στον Πίνακα \ref{tab:comp_no_repair}.

\begin{table}[H]
\centering
\caption{Αποτελέσματα αξιοπιστίας εξαρτημάτων (χωρίς επιδιόρθωση)}
\label{tab:comp_no_repair}
\begin{tabular}{@{}lcccc@{}}
\toprule
\textbf{Εξάρτημα} & \textbf{$R_{\text{exp}}$} & \textbf{$R_{\text{theo}}$} & \textbf{Σχετικό Σφάλμα (\%)} & \textbf{$\lambda_{\text{exp}}$ (fail/h)} \\ 
\midrule
C1 & 0.3410 & 0.3679 & 7.3 & 0.035862 \\
C2 & 0.0160 & 0.0155 & 3.2 & 0.041352 \\
C3 & 0.0100 & 0.0129 & 22.7 & 0.046052 \\
C4 & 0.0160 & 0.0155 & 3.2 & 0.041352 \\
C5 & 0.0230 & 0.0246 & 6.6 & 0.037723 \\
C6 & 0.0200 & 0.0281 & 28.9 & 0.039120 \\
C7 & 0.3020 & 0.2976 & 1.5 & 0.029933 \\
\bottomrule
\end{tabular}
\end{table}

\textbf{Θεωρητικός Υπολογισμός:}
\begin{equation}
R_i(t) = e^{-\lambda_i \cdot \text{DC}_i \cdot t} = e^{-\frac{\text{DC}_i \cdot t}{\text{MTTF}_i}}
\end{equation}

\begin{figure}[H]
\centering
\includegraphics[width=\textwidth]{results/no_repair/component_reliability.png}
\caption{Σύγκριση πειραματικής και θεωρητικής αξιοπιστίας και ρυθμού αστοχίας εξαρτημάτων}
\label{fig:comp_reliability}
\end{figure}

\subsection{Αξιοπιστία Συστήματος}

Τα αποτελέσματα για το σύστημα στο χρόνο $T_s = 30h$ παρουσιάζονται στον Πίνακα \ref{tab:sys_no_repair}.

\begin{table}[H]
\centering
\caption{Αποτελέσματα αξιοπιστίας συστήματος (χωρίς επιδιόρθωση)}
\label{tab:sys_no_repair}
\begin{tabular}{@{}lccc@{}}
\toprule
\textbf{Μέτρο} & \textbf{Πειραματική Τιμή} & \textbf{Θεωρητική Τιμή} & \textbf{Σχετικό Σφάλμα (\%)} \\ 
\midrule
$R_{\text{system}}(T_s=30h)$ & 0.1860 & 0.1811 & 2.7 \\
$\lambda_{\text{system}}$ (fail/h) & 0.056067 & 0.056961 & -- \\
MTTF$_{\text{system}}$ (h) & 13.97 & -- & -- \\
\bottomrule
\end{tabular}
\end{table}

\textbf{Θεωρητικός Υπολογισμός:}

Υπολογίζουμε πρώτα την αξιοπιστία κάθε μπλοκ:
\begin{align}
R_{\text{block2}} &= 1 - (1-R_{C2})(1-R_{C3})(1-R_{C4}) \\
R_{\text{block3}} &= 1 - (1-R_{C5})(1-R_{C6})
\end{align}

Η αξιοπιστία του συστήματος (σύνδεση σε σειρά):
\begin{equation}
R_{\text{system}} = R_{C1} \cdot R_{\text{block2}} \cdot R_{\text{block3}} \cdot R_{C7}
\end{equation}

\begin{figure}[H]
\centering
\includegraphics[width=\textwidth]{results/no_repair/system_reliability.png}
\caption{Αξιοπιστία συστήματος και κατανομή χρόνων αστοχίας}
\label{fig:sys_reliability}
\end{figure}

\begin{figure}[H]
\centering
\includegraphics[width=\textwidth]{results/no_repair/timeline_no_repair.png}
\caption{Χρονοδιάγραμμα δείγματος προσομοίωσης - Καταστάσεις συστήματος και εξαρτημάτων}
\label{fig:timeline_no_repair}
\end{figure}

\section{Αποτελέσματα - Ανάλυση με Επιδιόρθωση}

\subsection{Μετρικές Εξαρτημάτων}

Τα αποτελέσματα για τα εξαρτήματα με επιδιόρθωση παρουσιάζονται στον Πίνακα \ref{tab:comp_repair}.

\begin{table}[H]
\centering
\caption{Μετρικές εξαρτημάτων με επιδιόρθωση}
\label{tab:comp_repair}
\begin{tabular}{@{}lccccc@{}}
\toprule
\textbf{Εξάρτημα} & \textbf{MTBF$_{\text{exp}}$ (h)} & \textbf{MTBF$_{\text{theo}}$ (h)} & \textbf{MTTR$_{\text{exp}}$ (h)} & \textbf{$A_{\text{exp}}$} & \textbf{$A_{\text{theo}}$} \\ 
\midrule
C1 & 49.86 & 100.00 & 12.64 & 0.8320 & 0.8929 \\
C2 & 19.60 & 24.00 & 12.01 & 0.6941 & 0.6667 \\
C3 & 18.70 & 23.00 & 11.83 & 0.6824 & 0.6571 \\
C4 & 19.05 & 24.00 & 10.32 & 0.7147 & 0.7059 \\
C5 & 21.56 & 27.00 & 10.09 & 0.7498 & 0.7297 \\
C6 & 22.13 & 28.00 & 8.29 & 0.7875 & 0.7778 \\
C7 & 44.82 & 82.50 & 12.27 & 0.7850 & 0.8730 \\
\bottomrule
\end{tabular}
\end{table}

\textbf{Θεωρητικοί Υπολογισμοί:}
\begin{align}
\text{MTBF}_{\text{theo}} &= \frac{\text{MTTF}}{\text{DC}} \\
A_{\text{theo}} &= \frac{\text{MTBF}}{\text{MTBF} + \text{MTTR}}
\end{align}

\textbf{Σημείωση:} Η διαθεσιμότητα (A) υπολογίζεται ως το ποσοστό του χρόνου που το εξάρτημα δεν είναι υπό επιδιόρθωση (καταστάσεις 1 και 2). Για εξαρτήματα με DC < 1, η κατάσταση 1 (non-operational due to duty cycle) θεωρείται διαθέσιμη καθώς το εξάρτημα δεν έχει αστοχήσει, απλώς δεν απαιτείται να λειτουργεί.

\begin{figure}[H]
\centering
\includegraphics[width=\textwidth]{results/with_repair/component_repair.png}
\caption{Μετρικές αξιοπιστίας εξαρτημάτων με επιδιόρθωση}
\label{fig:comp_repair}
\end{figure}

\subsection{Μετρικές Συστήματος}

Τα αποτελέσματα για το σύστημα με επιδιόρθωση παρουσιάζονται στον Πίνακα \ref{tab:sys_repair}.

\begin{table}[H]
\centering
\caption{Μετρικές συστήματος με επιδιόρθωση}
\label{tab:sys_repair}
\begin{tabular}{@{}lc@{}}
\toprule
\textbf{Μέτρο} & \textbf{Τιμή} \\ 
\midrule
MTBF$_{\text{system}}$ (h) & 19.05 \\
MUT$_{\text{system}}$ (h) & 19.05 \\
MTTR$_{\text{system}}$ (h) & 8.11 \\
$A_{\text{system}}$ & 0.7409 (74.09\%) \\
$\lambda_{\text{system}}$ (fail/h) & 0.0525 \\
\bottomrule
\end{tabular}
\end{table}

\begin{figure}[H]
\centering
\includegraphics[width=\textwidth]{results/with_repair/system_repair.png}
\caption{Κατανομές MTBF και MTTR συστήματος και μετρικές αξιοπιστίας}
\label{fig:sys_repair}
\end{figure}

\begin{figure}[H]
\centering
\includegraphics[width=\textwidth]{results/with_repair/timeline_with_repair.png}
\caption{Χρονοδιάγραμμα συστήματος και εξαρτημάτων με επιδιόρθωση}
\label{fig:timeline_repair}
\end{figure}

\section{Ανάλυση Αποτελεσμάτων}

\subsection{Αξιοπιστία χωρίς Επιδιόρθωση}

\begin{itemize}
    \item Τα πειραματικά αποτελέσματα συμφωνούν καλά με τις θεωρητικές προβλέψεις (σφάλμα < 10\% για τα περισσότερα εξαρτήματα)
    \item Τα εξαρτήματα C3 και C6 παρουσιάζουν μεγαλύτερα σφάλματα (22.7\% και 28.9\%) λόγω του μικρού αριθμού δειγμάτων αστοχίας (μόνο 10 και 20 αστοχίες αντίστοιχα)
    \item Η αξιοπιστία του συστήματος είναι σημαντικά χαμηλότερη από αυτή των επιμέρους εξαρτημάτων, όπως αναμένεται για σύνδεση σε σειρά
    \item Το σύστημα έχει MTTF ≈ 14h, πολύ χαμηλότερο από τα MTTF των εξαρτημάτων
\end{itemize}

\subsection{Μετρικές με Επιδιόρθωση}

\begin{itemize}
    \item Το MTBF είναι μικρότερο από το θεωρητικό για τα εξαρτήματα C1 και C7 (με DC < 1), πιθανώς λόγω της αλληλεπίδρασης του duty cycle με τη διαδικασία επιδιόρθωσης
    \item Ο πειραματικός MTTR συμφωνεί πολύ καλά με το θεωρητικό (σφάλμα < 10\%)
    \item Η διαθεσιμότητα του συστήματος είναι 74.09\%, που σημαίνει ότι το σύστημα λειτουργεί κανονικά περίπου 3/4 του χρόνου
    \item Το σύστημα έχει MTBF ≈ 19h και MTTR ≈ 8h
\end{itemize}

\section{Περιγραφή Κώδικα}

Ο κώδικας οργανώνεται σε δύο κύρια αρχεία:

\subsection{simulation\_no\_repair.py}

Υλοποιεί την προσομοίωση Monte Carlo χωρίς επιδιόρθωση:
\begin{itemize}
    \item \texttt{simulate\_component()}: Προσομοιώνει ένα εξάρτημα μέχρι την πρώτη αστοχία
    \item \texttt{simulate\_system()}: Προσομοιώνει το σύστημα με τη λογική των παράλληλων μπλοκ
    \item \texttt{run\_component\_analysis()}: Εκτελεί N προσομοιώσεις για κάθε εξάρτημα
    \item \texttt{run\_system\_analysis()}: Εκτελεί N προσομοιώσεις για το σύστημα
\end{itemize}

Υπολογίζει: $\lambda$, $R(T_c)$, MTTF

\subsection{simulation\_with\_repair.py}

Υλοποιεί την προσομοίωση Monte Carlo με επιδιόρθωση:
\begin{itemize}
    \item \texttt{simulate\_component()}: Προσομοιώνει εξάρτημα με δυνατότητα επιδιόρθωσης
    \item \texttt{simulate\_system()}: Προσομοιώνει σύστημα με δυνατότητα επιδιόρθωσης εξαρτημάτων
    \item Καταγράφει όλες τις περιόδους λειτουργίας και επιδιόρθωσης
\end{itemize}

Υπολογίζει: MTBF, MUT, MTTR, A

\subsection{Κύριες Παράμετροι}

\begin{lstlisting}
Tc = 100.0      # Χρόνος μελέτης εξαρτημάτων
Ts = 30.0       # Χρόνος μελέτης συστήματος
DT = 0.01       # Χρονικό βήμα
N_SIMS = 1000   # Αριθμός προσομοιώσεων Monte Carlo
\end{lstlisting}

\section{Συμπεράσματα}

\begin{enumerate}
    \item Η μέθοδος Monte Carlo παρέχει ακριβείς εκτιμήσεις των μετρικών αξιοπιστίας, με σφάλματα κάτω από 10\% για τα περισσότερα μεγέθη
    \item Η αξιοπιστία του συστήματος επηρεάζεται σημαντικά από τη σύνδεση σε σειρά των εξαρτημάτων
    \item Η επιδιόρθωση βελτιώνει δραστικά τη διαθεσιμότητα του συστήματος, επιτρέποντας λειτουργία για μεγαλύτερες περιόδους
    \item Το duty cycle επηρεάζει τόσο την αξιοπιστία όσο και τη διαθεσιμότητα των εξαρτημάτων
    \item Το σύστημα με επιδιόρθωση επιτυγχάνει διαθεσιμότητα 74\%, που είναι αποδεκτή για πολλές πρακτικές εφαρμογές
\end{enumerate}

\end{document}
