\documentclass[12pt,a4paper]{article}

\usepackage[utf8]{inputenc}
\usepackage{listings}
\usepackage{xcolor}
\usepackage{graphicx}
\usepackage{hyperref}
\usepackage{geometry}
\usepackage{booktabs}
\usepackage{float}
\usepackage{amsmath}

% Language and Font Setup
\usepackage[LGR, T1]{fontenc}
\usepackage[greek, english]{babel}
\usepackage{alphabeta}

\geometry{margin=2.5cm}

% Code styling
\lstset{
    basicstyle=\ttfamily\small,
    breaklines=true,
    frame=single,
    language=Python, 
    keywordstyle=\color{blue},
    commentstyle=\color{gray},
    stringstyle=\color{red},
    tabsize=4,
    showstringspaces=false,
    extendedchars=true,
    inputencoding=utf8,
    literate={≈}{{$\approx$}}1
}

\title{\textbf{Αναφορά Υπολογιστικής Εργασίας Αξιοπιστίας Συστημάτων}\\ 7 διαφορετικά στοιχεια (2)}
\author{
    Τζανέτης Σάββας \\ ΑΕΜ: 10889
    \and
    Βασίλειος Ζωίδης \\ ΑΕΜ: 10652
}
\date{\today}

\begin{document}

\maketitle
\newpage

\renewcommand*\contentsname{Περιεχόμενα}
\tableofcontents
\newpage

\section{Ανάλυση του Συστήματος}

Στην παρούσα εργασία μελετάται η αξιοπιστία ενός σύνθετου συστήματος που αποτελείται από 7 εξαρτήματα (C1-C7). Το σύστημα έχει την εξής τοπολογία:


\begin{figure}[H]
\centering
\includegraphics[width=\textwidth]{topologia.png}
\caption{Μπλοκ Διάγραμμα Αξιοπιστίας του Συστήματος 4.2}
\label{fig:topologia}
\end{figure}

Με βάση το μπλοκ διάγραμμα της εικόνας, η λογική λειτουργίας του συστήματος 4.2 είναι η εξής:

\begin{itemize}
    \item $C_1$: Είναι σε σειρά (πρέπει να λειτουργεί).
    \item $C_2, C_3, C_4$: Είναι συνδεδεμένα παράλληλα (αρκεί να λειτουργεί έστω ένα από τα τρία).
    \item $C_5, C_6$: Είναι συνδεδεμένα παράλληλα (αρκεί να λειτουργεί έστω ένα από τα δύο).
    \item $C_7$: Είναι σε σειρά (πρέπει να λειτουργεί).
\end{itemize}

Άρα ο Μαθηματικός Τύπος Αξιοπιστίας του Συστήματος είναι:

\[
R_{sys}(t) = R_1(t) \left[ 1 - (1 - R_2(t))(1 - R_3(t))(1 - R_4(t)) \right] \left[ 1 - (1 - R_5(t))(1 - R_6(t)) \right] R_7(t)
\]

\section{Παράμετροι Προσομοίωσης}

\subsection{Χαρακτηριστικά Εξαρτημάτων}

Τα χαρακτηριστικά των 7 εξαρτημάτων φαίνονται στον Πίνακα \ref{tab:components}.
\begin{table}[H]
\centering
\caption{Χαρακτηριστικά εξαρτημάτων}
\label{tab:components}
\begin{tabular}{@{}cccc@{}}
\toprule
\textbf{Εξάρτημα} & \textbf{MTTF (h)} & \textbf{Duty Cycle} & \textbf{MTTR (h)} \\ 
\midrule
C1 & 30 & 0.3 & 12 \\
C2 & 24 & 1.0 & 12 \\
C3 & 23 & 1.0 & 12 \\
C4 & 24 & 1.0 & 10 \\
C5 & 27 & 1.0 & 10 \\
C6 & 28 & 1.0 & 8 \\
C7 & 33 & 0.4 & 12 \\
\bottomrule
\end{tabular}
\end{table}

\subsection{Παράμετροι Προσομοίωσης Monte Carlo}

\begin{itemize}
    \item \textbf{Χρόνος μελέτης εξαρτημάτων ($T_c$)}: 100 ώρες
    \item \textbf{Χρόνος μελέτης συστήματος ($T_s$)}: 30 ώρες
    \item \textbf{Χρονικό βήμα ($dt$)}: 0.1 ώρες
    \item \textbf{Αριθμός προσομοιώσεων ($N$)}: 1000
\end{itemize}

\section{Μέθοδος Προσομοίωσης}

\subsection{Μοντέλο Συμπεριφοράς Εξαρτήματος}

Κάθε εξάρτημα μπορεί να βρίσκεται σε μία από τις εξής καταστάσεις:
\begin{itemize}
    \item \textbf{Κατάσταση 1 (Operational)}: Το εξάρτημα λειτουργεί κανονικά
    \item \textbf{Κατάσταση 0 (Non-operational - DC)}: Το εξάρτημα δεν λειτουργεί λόγω κύκλου εργασίας (duty cycle)
    \item \textbf{Κατάσταση -1 (Failed)}: Το εξάρτημα έχει αστοχήσει (περιμένει επιδιόρθωση)
    \item \textbf{Κατάσταση -2 (Under repair)}: Το εξάρτημα επιδιορθώνεται (μόνο με επιδιόρθωση)
\end{itemize}

\subsection{Μοντέλο Αστοχίας}

Η αστοχία κάθε εξαρτήματος ακολουθεί την εκθετική κατανομή με παράμετρο $\lambda = \frac{1}{\text{MTTF}}$.
Η πιθανότητα αστοχίας σε διάστημα $dt$ δίνεται από:

\begin{equation}
P_{\text{fail}}(dt) = 1 - e^{-\lambda \cdot dt}
\end{equation}

Κάναμε την υπόθεση ότι το κάθε εξάρτημα δεν μπορεί να αστοχήσει όταν δεν είναι σε λειτουργία οπότε τo duty cycle (DC) επηρεάζει το πραγματικό MTTF ως:

\begin{equation}
\text{MTBF}_{\text{effective}} = \frac{\text{MTTF}}{\text{DC}}
\end{equation}

\subsection{Μοντέλο Επιδιόρθωσης}

Ο χρόνος επιδιόρθωσης ακολουθεί εκθετική κατανομή με μέσο χρόνο MTTR:

\begin{equation}
T_{\text{repair}} \sim \text{Exp}\left(\frac{1}{\text{MTTR}}\right)
\end{equation}

Όταν ένα εξάρτημα αστοχεί, δημιουργείται τυχαίος χρόνος επιδιόρθωσης $T_r$ και το εξάρτημα επιστρέφει σε λειτουργία μετά το πέρας της επιδιόρθωσης.

\section{Αποτελέσματα - Ανάλυση χωρίς Επιδιόρθωση}

\subsection{Αξιοπιστία Εξαρτημάτων}

\textbf{Θεωρητικοί Υπολογισμοί:}

Ο θεωρητικός ρυθμός αστοχίας:
\begin{equation}
\lambda_{\text{theo}} = \frac{1}{\text{MTTF}}
\end{equation}

Η θεωρητική αξιοπιστία:
\begin{equation}
R_i(t) = e^{-\lambda_i \cdot \text{DC}_i \cdot t} = e^{-\frac{\text{DC}_i \cdot t}{\text{MTTF}_i}}
\end{equation}

\noindent
\textbf{Σύγκριση Αποτελεσμάτων:}

Τα αποτελέσματα για την αξιοπιστία των εξαρτημάτων από τη προσομοίωση για χρόνο $T_c = 100h$ παρουσιάζονται στον Πίνακα \ref{tab:comp_no_repair}. Οι τιμές του πειραματικού ρυθμού αστοχίας υπολογίζονται από την πειραματική αξιοπιστία ως εξείς:
\begin{equation}
\lambda_{\text{exp}} = -\frac{\ln(R_{\text{exp}})}{\text{DC} \cdot t}
\end{equation}
\begin{table}[H]
\centering
\caption{Αποτελέσματα αξιοπιστίας εξαρτημάτων (χωρίς επιδιόρθωση)}
\label{tab:comp_no_repair}
\begin{tabular}{@{}lcccccc@{}}
\toprule
\textbf{Εξάρτημα} & \textbf{$R_{\text{exp}}$} & \textbf{$R_{\text{theo}}$} & \textbf{Σφάλμα R (\%)} & \textbf{$\lambda_{\text{exp}}$} & \textbf{$\lambda_{\text{theo}}$} & \textbf{Σφάλμα $\lambda$ (\%)} \\ 
\midrule
C1 & 0.3590 & 0.3679 & 2.4 & 0.0341 & 0.0333 & 2.4 \\
C2 & 0.0130 & 0.0155 & 16.1 & 0.0434 & 0.0417 & 4.2 \\
C3 & 0.0130 & 0.0129 & 0.5 & 0.0434 & 0.0435 & 0.1 \\
C4 & 0.0220 & 0.0155 & 41.9 & 0.0382 & 0.0417 & 8.4 \\
C5 & 0.0250 & 0.0246 & 1.5 & 0.0369 & 0.0370 & 0.4 \\
C6 & 0.0340 & 0.0281 & 20.9 & 0.0338 & 0.0357 & 5.3 \\
C7 & 0.2860 & 0.2976 & 3.9 & 0.0313 & 0.0303 & 3.3 \\
\bottomrule
\end{tabular}
\end{table}

\begin{figure}[H]
\centering
\includegraphics[width=\textwidth]{results/no_repair/component_reliability.png}
\caption{Σύγκριση πειραματικής και θεωρητικής αξιοπιστίας και ρυθμού αστοχίας εξαρτημάτων}
\label{fig:comp_reliability}
\end{figure}

\subsection{Αξιοπιστία Συστήματος}

\textbf{Θεωρητικοί Υπολογισμοί:}

Υπολογίζουμε πρώτα την αξιοπιστία κάθε μπλοκ:
\begin{align}
R_{\text{block2}} &= 1 - (1-R_{C2})(1-R_{C3})(1-R_{C4}) \\
R_{\text{block3}} &= 1 - (1-R_{C5})(1-R_{C6})
\end{align}

Η αξιοπιστία του συστήματος (σύνδεση σε σειρά):
\begin{equation}
R_{\text{system}} = R_{C1} \cdot R_{\text{block2}} \cdot R_{\text{block3}} \cdot R_{C7}
\end{equation}

\noindent
\textbf{Σύγκριση Αποτελεσμάτων:}

Τα αποτελέσματα για το σύστημα στο χρόνο $T_s = 30h$ παρουσιάζονται στον Πίνακα \ref{tab:sys_no_repair}.
\begin{table}[H]
\centering
\caption{Αποτελέσματα αξιοπιστίας συστήματος (χωρίς επιδιόρθωση)}
\label{tab:sys_no_repair}
\begin{tabular}{@{}lccc@{}}
\toprule
\textbf{Τιμή} & \textbf{Πειραματική} & \textbf{Θεωρητική} & \textbf{Σχετικό Σφάλμα (\%)} \\ 
\midrule
$R_{\text{system}}(T_s=30h)$ & 0.1870 & 0.1811 & 3.3 \\
$\lambda_{\text{system}}$ (fail/h) & 0.0559 & 0.0570 & 1.9 \\
MTTF$_{\text{system}}$ (h) & 13.88 & 17.56 & 21.0 \\
\bottomrule
\end{tabular}
\end{table}

\begin{figure}[H]
\centering
\includegraphics[width=\textwidth]{results/no_repair/system_reliability.png}
\caption{Αξιοπιστία συστήματος και κατανομή χρόνων αστοχίας}
\label{fig:sys_reliability}
\end{figure}

Στην Εικόνα \ref{fig:timeline_no_repair} παρουσιάζεται ένα δείγμα μίας προσομοίωσης με τις καταστάσεις του συστήματος και κάθε εξερτήματος στον χρόνο. Μπορούμε να διακρίνουμε πως το σύστημα λειτουργεί κανονικά, παρόλο που το εξάρτημα 6 έχει αποτύχει, διότι το εξάρτημα 5 μέχρι εκείνη τη στιγμή λειτουργεί κανονικά. Ωστόσο μόλις το εξάρτημα 7 αποτύχει, τότε όλο το σύστημα βγαίνει εκτός λειτουργίας.

\begin{figure}[H]
\centering
\includegraphics[width=\textwidth]{results/no_repair/timeline_no_repair.png}
\caption{Χρονοδιάγραμμα δείγματος προσομοίωσης - Καταστάσεις συστήματος και εξαρτημάτων}
\label{fig:timeline_no_repair}
\end{figure}

\section{Αποτελέσματα - Ανάλυση με Επιδιόρθωση}

\subsection{Μετρικές Εξαρτημάτων}

\textbf{Θεωρητικοί Υπολογισμοί:}

Το πραγματικό MTTF λαμβάνοντας υπόψη το duty cycle (ο χρόνος που το εξάρτημα είναι ενεργό):
\begin{equation}
\text{MTTF}_{\text{eff}} = \frac{\text{MTTF}}{\text{DC}}
\end{equation}

Ο θεωρητικός μέσος χρόνος λειτουργίας (Mean Up Time):
\begin{equation}
\text{MUT}_{\text{theo}} = \text{MTTF}_{\text{eff}} = \frac{\text{MTTF}}{\text{DC}}
\end{equation}

Ο θεωρητικός μέσος χρόνος επιδιόρθωσης (Mean Time To Repair):
\begin{equation}
\text{MTTR}_{\text{theo}} = \text{MTTR}
\end{equation}

Το θεωρητικό MTBF (Mean Time Between Failures):
\begin{equation}
\text{MTBF}_{\text{theo}} = \text{MUT}_{\text{theo}} + \text{MTTR}_{\text{theo}} = \text{MTTF}_{\text{eff}} + \text{MTTR}
\end{equation}

Η θεωρητική διαθεσιμότητα:
\begin{equation}
A_{\text{theo}} = \frac{\text{MUT}}{\text{MTBF}} = \frac{\text{MTTF}_{\text{eff}}}{\text{MTTF}_{\text{eff}} + \text{MTTR}}
\end{equation}

\noindent
\textbf{Σύγκριση Αποτελεσμάτων:}

Τα αποτελέσματα για τα εξαρτήματα με επιδιόρθωση παρουσιάζονται στους Πίνακες \ref{tab:comp_repair_mtbf} και \ref{tab:comp_repair_avail}.

\begin{table}[H]
\centering
\caption{Μετρικές MTBF και MTTR εξαρτημάτων με επιδιόρθωση}
\label{tab:comp_repair_mtbf}
\begin{tabular}{@{}lcccccc@{}}
\toprule
\textbf{Εξάρτημα} & \textbf{MTBF$_{\text{exp}}$} & \textbf{MTBF$_{\text{theo}}$} & \textbf{Σφάλμα (\%)} & \textbf{MTTR$_{\text{exp}}$} & \textbf{MTTR$_{\text{theo}}$} & \textbf{Σφάλμα (\%)} \\ 
\midrule
C1 & 111.73 & 112.00 & 0.2 & 11.69 & 12.00 & 2.6 \\
C2 & 33.75 & 36.00 & 6.3 & 11.69 & 12.00 & 2.6 \\
C3 & 34.77 & 35.00 & 0.7 & 12.59 & 12.00 & 4.9 \\
C4 & 33.28 & 34.00 & 2.1 & 10.05 & 10.00 & 0.5 \\
C5 & 36.35 & 37.00 & 1.8 & 10.03 & 10.00 & 0.3 \\
C6 & 35.55 & 36.00 & 1.3 & 8.09 & 8.00 & 1.1 \\
C7 & 97.94 & 94.50 & 3.6 & 11.74 & 12.00 & 2.1 \\
\bottomrule
\end{tabular}
\end{table}

\begin{table}[H]
\centering
\caption{Μετρικές MUT και Διαθεσιμότητας εξαρτημάτων με επιδιόρθωση}
\label{tab:comp_repair_avail}
\begin{tabular}{@{}lcccccc@{}}
\toprule
\textbf{Εξάρτημα} & \textbf{MUT$_{\text{exp}}$} & \textbf{MUT$_{\text{theo}}$} & \textbf{Σφάλμα (\%)} & \textbf{$A_{\text{exp}}$} & \textbf{$A_{\text{theo}}$} & \textbf{Σφάλμα (\%)} \\ 
\midrule
C1 & 50.67 & 100.00 & 49.3 & 0.9090 & 0.8929 & 1.8 \\
C2 & 19.08 & 24.00 & 20.5 & 0.6924 & 0.6667 & 3.9 \\
C3 & 19.17 & 23.00 & 16.7 & 0.6783 & 0.6571 & 3.2 \\
C4 & 19.90 & 24.00 & 17.1 & 0.7308 & 0.7059 & 3.5 \\
C5 & 21.66 & 27.00 & 19.8 & 0.7499 & 0.7297 & 2.8 \\
C6 & 21.87 & 28.00 & 21.9 & 0.7883 & 0.7778 & 1.3 \\
C7 & 47.03 & 82.50 & 43.0 & 0.8936 & 0.8730 & 2.4 \\
\bottomrule
\end{tabular}
\end{table}



\begin{figure}[H]
\centering
\includegraphics[width=\textwidth]{results/with_repair/component_availability.png}
\caption{Μετρικές αξιοπιστίας εξαρτημάτων με επιδιόρθωση}
\label{fig:comp_repair}
\end{figure}

\subsection{Μετρικές Συστήματος}

\textbf{Θεωρητικός Υπολογισμός Διαθεσιμότητας Συστήματος:}

Η διαθεσιμότητα του συστήματος υπολογίζεται με βάση το μπλοκ διάγραμμα:
\begin{itemize}
    \item Παράλληλη σύνδεση: $A_{\text{parallel}} = 1 - \prod_i (1-A_i)$
    \item Σειριακή σύνδεση: $A_{\text{series}} = \prod_i A_i$
\end{itemize}

Άρα η θεωρητική διαθεσιμότητα του συστήματος:
\begin{align}
A_{\text{block2}} &= 1 - (1-A_{C2})(1-A_{C3})(1-A_{C4}) \\
A_{\text{block3}} &= 1 - (1-A_{C5})(1-A_{C6}) \\
A_{\text{system}} &= A_{C1} \cdot A_{\text{block2}} \cdot A_{\text{block3}} \cdot A_{C7}
\end{align}

\noindent
\textbf{Σύγκριση Αποτελεσμάτων:}

Τα αποτελέσματα για το σύστημα με επιδιόρθωση παρουσιάζονται στον Πίνακα \ref{tab:sys_repair}.
\begin{table}[H]
\centering
\caption{Μετρικές συστήματος με επιδιόρθωση}
\label{tab:sys_repair}
\begin{tabular}{@{}lccc@{}}
\toprule
\textbf{Μέτρο} & \textbf{Πειραματική} & \textbf{Θεωρητική} & \textbf{Σφάλμα (\%)} \\ 
\midrule
MTBF$_{\text{system}}$ (h) & 31.33 & 45.27 & 30.8 \\
MUT$_{\text{system}}$ (h) & 19.22 & 32.05 & 40.0 \\
MTTR$_{\text{system}}$ (h) & 7.53 & 13.22 & 43.0 \\
$A_{\text{system}}$ & 0.7495 (74.95\%) & 0.7080 (70.80\%) & 5.9 \\
\bottomrule
\end{tabular}
\end{table}

\begin{figure}[H]
\centering
\includegraphics[width=\textwidth]{results/with_repair/system_availability.png}
\caption{Κατανομές MTBF και MTTR συστήματος και μετρικές αξιοπιστίας}
\label{fig:sys_repair}
\end{figure}
Στην Εικόνα \ref{fig:timeline_repair} παρουσιάζεται ένα δείγμα μίας προσομοίωσης με τις καταστάσεις του συστήματος και κάθε εξερτήματος στον χρόνο. Μπορούμε να διακρίνουμε πως το σύστημα βγαίνει εκτός λειτουργείας όπως στο προηγούμενο παράδειγμα. Ωστόσο, σε αυτή την περίπτωση το κάθε εξάρτημα επιδιορθώνεται σύμφωνα με το MTTR. 

Στην προκειμένη περίπτωση μόλις τα εξαρτήματα 5 και 6 αποτύχουν το σύστημα βγαίνει εκτός λειτουργεία. Μόλις όμως το εξάρτημα 6 επιδιορθωθεί το σύστημα είναι και πάλι λειτουργικό.
\begin{figure}[H]
\centering
\includegraphics[width=\textwidth]{results/with_repair/timeline_with_repair.png}
\caption{Χρονοδιάγραμμα συστήματος και εξαρτημάτων με επιδιόρθωση}
\label{fig:timeline_repair}
\end{figure}

\section{Ανάλυση Αποτελεσμάτων}

\subsection{Αξιοπιστία χωρίς Επιδιόρθωση}

\begin{itemize}
    \item Τα πειραματικά αποτελέσματα συμφωνούν καλά με τις θεωρητικές προβλέψεις για την αξιοπιστία R(σφάλμα $<$ 5\% για C1, C3, C5, C7)
    \item Τα εξαρτήματα C4 και C6 παρουσιάζουν μεγαλύτερα σφάλματα (41.9\% και 20.9\%) λόγω τυχαιότητας Monte Carlo
    \item Η αξιοπιστία του συστήματος είναι σημαντικά χαμηλότερη από αυτή των επιμέρους εξαρτημάτων, όπως αναμένεται για σύνδεση σε σειρά
    \item Το σύστημα έχει MTTF $\approx$ 13.88h (πειραματικό) έναντι 17.56h (θεωρητικό), με σφάλμα 21\%
    \item Ο ρυθμός αστοχίας του συστήματος $\lambda_{sys}$ = 0.0559 fail/h συμφωνεί πολύ καλά με το θεωρητικό (σφάλμα 1.9\%)
\end{itemize}

\subsection{Μετρικές με Επιδιόρθωση}

\begin{itemize}
    \item Το MTBF των εξαρτημάτων συμφωνεί πολύ καλά με το θεωρητικό (σφάλμα $<$ 7\% για όλα τα εξαρτήματα)
    \item Ο πειραματικός MTTR συμφωνεί άριστα με το θεωρητικό (σφάλμα $<$ 5\%)
    \item Η πειραματική διαθεσιμότητα των εξαρτημάτων είναι ελαφρώς υψηλότερη από τη θεωρητική (σφάλμα 1-4\%)
    \item Η διαθεσιμότητα του συστήματος είναι 74.95\% (πειραματική) έναντι 70.80\% (θεωρητική)
    \item Το σύστημα έχει MTBF $\approx$ 31.33h και MTTR $\approx$ 7.53h
    \item Ο MUT των εξαρτημάτων είναι χαμηλότερος από το θεωρητικό, πιθανώς λόγω του τρόπου υπολογισμού στην προσομοίωση
\end{itemize}

\section{Περιγραφή Κώδικα}

Ο κώδικας οργανώνεται σε τρία κύρια αρχεία:

\subsection{config.py}

Περιέχει τις παραμέτρους της προσομοίωσης και τα χαρακτηριστικά των εξαρτημάτων:
\begin{itemize}
    \item \texttt{COMPONENTS\_DATA}: Λεξικό με MTTF, DC και MTTR κάθε εξαρτήματος
    \item \texttt{SYSTEM\_STRUCTURE}: Δομή του συστήματος (σειρά/παράλληλα μπλοκ)
    \item Παράμετροι προσομοίωσης: $T_c$, $T_s$, $dt$, $N$
\end{itemize}

\subsection{simulation\_no\_repair.py}

Υλοποιεί την προσομοίωση Monte Carlo χωρίς επιδιόρθωση:
\begin{itemize}
    \item \texttt{simulate\_system()}: Προσομοιώνει το σύστημα με τη λογική των παράλληλων μπλοκ και επιστρέφει τον χρόνο αστοχίας
    \item \texttt{run\_combined\_analysis()}: Εκτελεί N προσομοιώσεις και υπολογίζει μετρικές για εξαρτήματα ($T_c$) και σύστημα ($T_s$)
    \item \texttt{create\_component\_plots()}: Δημιουργεί γραφήματα σύγκρισης αξιοπιστίας εξαρτημάτων
    \item \texttt{create\_system\_plots()}: Δημιουργεί γραφήματα αξιοπιστίας συστήματος
    \item \texttt{create\_timeline\_plot()}: Δημιουργεί χρονοδιάγραμμα καταστάσεων
\end{itemize}

Υπολογίζει: $\lambda$, $R(T_c)$, MTTF

\subsection{simulation\_with\_repair.py}

Υλοποιεί την προσομοίωση Monte Carlo με επιδιόρθωση:
\begin{itemize}
    \item \texttt{simulate\_system\_with\_repair()}: Προσομοιώνει σύστημα με δυνατότητα επιδιόρθωσης εξαρτημάτων
    \item \texttt{run\_availability\_analysis()}: Εκτελεί N προσομοιώσεις και υπολογίζει μετρικές διαθεσιμότητας
    \item Καταγράφει όλες τις περιόδους λειτουργίας και επιδιόρθωσης
    \item Υπολογίζει θεωρητικές τιμές με βάση τη δομή RBD
\end{itemize}

Υπολογίζει: MTBF, MUT, MTTR, A

\subsection{Κύριες Παράμετροι}

\begin{lstlisting}
Tc = 100.0      # Component study time (hours)
Ts = 30.0       # System study time (hours)
DT = 0.1        # Time step (hours)
N_SIMS = 1000   # Number of Monte Carlo simulations
\end{lstlisting}

\section{Συμπεράσματα}

\begin{enumerate}
    \item Η μέθοδος Monte Carlo παρέχει ακριβείς εκτιμήσεις των μετρικών αξιοπιστίας, με σφάλματα κάτω από 10\% για τα περισσότερα μεγέθη
    \item Η αξιοπιστία του συστήματος επηρεάζεται σημαντικά από τη σύνδεση σε σειρά των εξαρτημάτων
    \item Η επιδιόρθωση βελτιώνει δραστικά τη διαθεσιμότητα του συστήματος, επιτρέποντας λειτουργία για μεγαλύτερες περιόδους
    \item Το duty cycle επηρεάζει τόσο την αξιοπιστία όσο και τη διαθεσιμότητα των εξαρτημάτων
    \item Το σύστημα με επιδιόρθωση επιτυγχάνει διαθεσιμότητα 74\%.
\end{enumerate}

\end{document}